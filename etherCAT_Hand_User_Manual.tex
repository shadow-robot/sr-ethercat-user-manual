\documentclass[12pt]{article}
\usepackage[pdftex]{graphicx}
\usepackage{hyperref}
\hypersetup{
    colorlinks,
    citecolor=black,
    filecolor=black,
    linkcolor=black,
    urlcolor=cyan
}
\usepackage[text={16cm,24cm}]{geometry}
\usepackage{pgffor}
\usepackage{filecontents}% Used so that the external files can be placed in this file
\setlength{\parskip}{\baselineskip}
\usepackage{color}
\newcommand{\todo}[1]{\colorbox{yellow}{\textbf{#1}}}
\definecolor{shadowgrey}{rgb}{0.18,0.2,0.21}
\usepackage{listings}
\lstset{ %
  frame=shadowbox,
  rulesepcolor=\color{shadowgrey},
  basicstyle=\footnotesize,
  commentstyle=\footnotesize
}

\newcommand{\linuxtilde}{\raise.20ex\hbox{$\scriptstyle\mathtt{\sim}$}}


\title{\textbf{EtherCAT Dextrous Hand} \\
- User Manual -}
\author{Shadow Software team - software@shadowrobot.com}

\makeindex

\begin{document}
\begin{titlepage}

\maketitle
\vspace{5cm}
\begin{center}
\includegraphics{images/logo-shadowDB.png}
\end{center}
\end{titlepage}

\tableofcontents
\newpage

\section{Overview}
\label{sec:overview}

\par This user manual will get you started with using our etherCAT hand or simulated robots. We'll guide you through installing the code in the \hyperref[sec:install]{\textbf{section~\ref*{sec:install}}}, show you some basic commands and code samples in \hyperref[sec:running-our-code]{\textbf{section~\ref*{sec:running-our-code}}}, and explain where to find things in \hyperref[sec:navigate]{\textbf{section~\ref*{sec:navigate}}}. This guide ends with some pointers to essential documentation you should read (\hyperref[sec:where-to-go]{\textbf{section~\ref*{sec:where-to-go}}}), and some troubleshooting tips (\hyperref[sec:what-do-if]{\textbf{section~\ref*{sec:what-do-if}}}).

\par If you have any questions, don't hesitate to contact us at: \texttt{software@shadowrobot.com}.

\newpage

\section{Installing our Software}
\label{sec:install}
\par For a typical user, we really recommend installing our software directly from the apt repository (see section \ref{sec:install_apt}). This will ensure that you keep the latest released software on your computer.

\par We assume you've already installed ROS electric, as detailed on \href{http://www.ros.org/wiki/electric/Installation/Ubuntu}{the ROS wiki}.

\par However if you plan to modify some parts of our code, you can use an overlay to install some of the sources from our publicly available Launchpad repository on top of your existing installation.

\subsection{From our apt repository (recommended)}
\label{sec:install_apt}
\todo{Mark complete this section once we've setup the public apt-repo for our stacks}


\subsection{From sources}
\label{sec:install_src}
\par If you need to modify one of the packages for your own use, you'll want to first install the released version from the apt repository as explained in the section \ref{sec:install_src}. This will ensure that you have all the code you need. Once you've identified in which stack the package you want to modify is, you can create an overlay\footnote{\textbf{Overlay:} use another version of a stack than the default installed one.} to use the source version.

\par To create an overlay, we'll use the \href{http://ros.org/wiki/rosinstall}{rosinstall} utility. In this example, we'll overlay the \href{http://launchpad.net/sr-ros-interface}{shadow\_robot} stack, using trunk version of the code hosted on launchpad using \href{http://bazaar.canonical.com}{bzr}: \texttt{lp:sr-ros-interface}.

\begin{itemize}
\item First we need to install the rosinstall utility if you haven't done so already:
  \begin{lstlisting}[language=Bash]
> sudo apt-get install python-setuptools bzr
> sudo easy_install -U rosinstall vcstools
  \end{lstlisting}

\item We can now create a rosinstall file that will contain the information about the stack we want to overlay (in our case we're overlaying the shadow\_robot stack only):
  \begin{lstlisting}[title={\textbf{shadow\_robot.rosinstall}}][language=Python]
- bzr:                       # VCS used (bzr in our case)
  uri: 'lp:sr-ros-interface' # public uri for the code
  local-name: shadow_robot   # name under which we install the stack
  \end{lstlisting}

\item We're now going to create a workspace (in the \texttt{\linuxtilde/workspace} directory), based on your currently installed ROS system (we're assuming you're using ROS electric), and use the rosinstall file we created to download the code into it.
  \begin{lstlisting}[escapeinside=''][language=Bash]
> rosinstall '\linuxtilde'/workspace /opt/ros/electric/setup.bash
> rosinstall '\linuxtilde'/workspace shadow_robot.rosinstall
  \end{lstlisting}

\item To use the newly created environment, simply source the setup.bash created in the previous step (to have it sourced automatically when you open a new terminal, simply add it to your \texttt{\linuxtilde/.bashrc} as show on the second line).
  \begin{lstlisting}[escapeinside=''][language=Bash]
> source '\linuxtilde'/workspace/setup.bash
> echo source '\linuxtilde'/workspace/setup.bash >> '\linuxtilde'/.bashrc
  \end{lstlisting}

\item Finally, you can build the whole stack in one go using the \texttt{rosmake} command:
  \begin{lstlisting}[escapeinside=''][language=Bash]
> rosmake --rosdep-install --rosdep-yes shadow_robot
  \end{lstlisting}

\end{itemize}
\newpage

\section{Running our code}
\label{sec:running-our-code}

\subsection{Starting the real hand}
\label{sec:starting-real-hand}

\subsection{Starting the simulated robot}
\label{sec:start-simul-hand}

\subsection{Checking it works}
\label{sec:checking-it-works}

\newpage

\section{Navigating our code}
\label{sec:navigate}

\subsection{Where to find the settings}
\label{sec:where-find-settings}

\subsubsection{Specifying system parameters}
\label{sec:spec-system-param}

\subsubsection{Specifying default controllers settings}
\label{sec:spec-defa-contr}

\newpage

\section{Where to go next}
\label{sec:where-to-go}

\newpage

\section{What to do if it doesn't work}
\label{sec:what-do-if}

\newpage

\section{Changelog}
\label{sec:changelog}

\newcommand{\filename}[1]{changelogs/file#1}
\newcommand{\buildchangelog}
{%
  \foreach \i in {0, ...,99}%
  {%
    \expandafter\IfFileExists\expandafter{\expandafter\filename\expandafter\i}%
    {%
      \expandafter\input\expandafter{\filename\i}%
    }%
    {}%
  }%
}
\buildchangelog

\end{document}

